\documentclass[a4,12pt]{article}%
\usepackage{amsmath,amsfonts,amsthm}% ams packages
\usepackage[svgnames]{xcolor}% named colors
\usepackage[hang,small,labelfont=bf,up,textfont=it,up]{caption}% caption
\usepackage{booktabs}% tables

\title{Application: The Distribution of Wealth At the Very Top}
\author{Patrick Toche}
\date{}

\begin{document}

\maketitle

\newpage


\section*{Wealth Distributions}

The purpose of this project is to explore the persistence and transmission of wealth at the very top of the wealth distribution, using a ranking of the wealthiest individuals compiled by Forbes Magazine. Download the Forbes List for 1996 [xls] and Forbes List for 2014 [xls] on the world's wealthiest individuals.


\begin{enumerate}

\item 
Identify all the individuals who appear on both lists, and for each one record their wealth. Note that the names often appear in different forms (e.g. "Gates, William III" and "Bill Gates"), so you will need to be careful in matching the names. Using the Consumer Price Index for the United States, convert the 1996 reported wealth to 2014 dollars. Then, using these 2014-dollar values, compute the annualised growth rate for these individuals' wealth (the 'real' annual growth rates). Compare the highest and lowest growth rates you have computed. Comment on the results. 

Hint: think about what could account for the different growth rates observed; think about how to interpret the highest and the lowest observed growth rates; what does this data sample leave out?

For your convenience, a data file of the CPI for urban consumers in the United States is available [xls], as downloaded from Fred on 17 March 2015.

\item
Identify all individuals in the 2014 list who have inherited wealth from an individual in the 1996 list, e.g. Beate Heister \& Karl Albrecht Junior inherited from Karl Albrecht (now deceased), while Theo Albrecht Junior inherited from Theo Albrecht (now deceased). The 2014 list contains a column for the individual's age, so this may be used in identifying such instances. In several cases, children have the same surname as their parents, so another strategy is to look through the surnames and do a quick web search to find out if the death was reported in the news. In each case you have identified, record the number of children and widowed/divorced spouses among which the wealth was divided. What is the typical number of children? of spouses? Comment on the effect of death on wealth concentration. 

Hint: if a deceased had 20 children, the descendants would likely have dropped out of the list.

\item 
Rank individuals by their age in 2014 and for those individuals over 85 find out how many children they have. First, plot the distribution of wealth among all individuals on the list aged 85-100. Secondly, plot the distribution of wealth among their children, assuming each child receives an equal share of the parent's wealth (e.g. if David Rockefeller Sr has 6 children, each will inherit a sixth of their father's 3 billion, i.e. 0.5 billion). Compare the two distributions and comment.

Hint: To compare the distributions, you may find it useful to standardize the distributions and overlay them.

\end{enumerate}

You are encouraged to cooperate on identifying individuals and creating the lists, by sharing your findings with other students from other groups. However, please submit a unique piece of work for each group. Remember to put some thinking into the design of your graphs: are they informative? are they labeled? captioned? legible? proportioned? colorful? simple? pretty? I value explanations and clarity too.



\end{document}